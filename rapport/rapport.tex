\documentclass[preprint, 3p,
authoryear]{elsarticle} %review=doublespace preprint=single 5p=2 column
%%% Begin My package additions %%%%%%%%%%%%%%%%%%%

\usepackage[hyphens]{url}

  \journal{An awesome journal} % Sets Journal name

\usepackage{lineno} % add

\usepackage{graphicx}
%%%%%%%%%%%%%%%% end my additions to header

\usepackage[T1]{fontenc}
\usepackage{lmodern}
\usepackage{amssymb,amsmath}
\usepackage{ifxetex,ifluatex}
\usepackage{fixltx2e} % provides \textsubscript
% use upquote if available, for straight quotes in verbatim environments
\IfFileExists{upquote.sty}{\usepackage{upquote}}{}
\ifnum 0\ifxetex 1\fi\ifluatex 1\fi=0 % if pdftex
  \usepackage[utf8]{inputenc}
\else % if luatex or xelatex
  \usepackage{fontspec}
  \ifxetex
    \usepackage{xltxtra,xunicode}
  \fi
  \defaultfontfeatures{Mapping=tex-text,Scale=MatchLowercase}
  \newcommand{\euro}{€}
\fi
% use microtype if available
\IfFileExists{microtype.sty}{\usepackage{microtype}}{}
\usepackage[]{natbib}
\bibliographystyle{plainnat}

\usepackage{graphicx}
\ifxetex
  \usepackage[setpagesize=false, % page size defined by xetex
              unicode=false, % unicode breaks when used with xetex
              xetex]{hyperref}
\else
  \usepackage[unicode=true]{hyperref}
\fi
\hypersetup{breaklinks=true,
            bookmarks=true,
            pdfauthor={},
            pdftitle={La centralité dans les réseaux de collaboration étudiant ; une comparaison des similitudes avec les réseaux écologiques},
            colorlinks=false,
            urlcolor=blue,
            linkcolor=magenta,
            pdfborder={0 0 0}}

\setcounter{secnumdepth}{0}
% Pandoc toggle for numbering sections (defaults to be off)
\setcounter{secnumdepth}{0}


% tightlist command for lists without linebreak
\providecommand{\tightlist}{%
  \setlength{\itemsep}{0pt}\setlength{\parskip}{0pt}}






\begin{document}


\begin{frontmatter}

  \title{La centralité dans les réseaux de collaboration étudiant ; une
comparaison des similitudes avec les réseaux écologiques}
    \author[a]{Philippe Bourassa%
  %
  }
   \ead{Philippe.Bourassa4@USherbrooke.ca} 
    \author[a]{Simon Bourgeois%
  %
  }
   \ead{Simon.Bourgeois@USherbrooke.ca} 
    \author[a]{Justine Labelle%
  %
  }
   \ead{Justine.Labelle@USherbrooke.ca} 
    \author[a]{Kayla Trempe-Kay%
  %
  }
   \ead{Kayla.Trempe-Kay@USherbrooke.ca} 
      \affiliation[a]{Universite de Sherbrooke, Departement de biologie,
2500 Boulevard de l'Universite, Sherbrooke, Quebec, J1K 2R1}
    \cortext[cor1]{Corresponding author}
  
  \begin{abstract}
  Les propriétés des réseaux écologiques fournissent des indications sur
  les interactions écologiques et la résilience du réseau face aux
  changements et aux perturbations. Parmi ces propriétés, la centralité
  a été utilisée pour comparer les similarités d'un réseau d'interaction
  entre les étudiants du cours BIO500 et un réseau écologique. Il a été
  possible d'identifier les étudiants avec les plus grandes centralités
  et de les associer à des espèces ``clés de voûte'' et généralistes. De
  plus, les étudiants qui ne sont pas visés par l'analyse du réseau de
  collaboration des étudiants du cours ont tendance à avoir une
  centralité plus faible. Nous avons aussi tenté d'observer l'influence
  de la session de début de baccalauréat et la formation préalable des
  étudiants sur la centralité. Toutefois, aucun de ces facteurs ne
  l'impact de manière significative, entre autres dû à la faible taille
  d'échantillon.
  \end{abstract}
    \begin{keyword}
    Réseau écologique \sep Collaboration \sep 
    Centralité
  \end{keyword}
  
 \end{frontmatter}

\hypertarget{introduction}{%
\section{Introduction}\label{introduction}}

Les écologistes ont comme pratique l'étude des interactions entre les
espèces qui peut être traduite et analysée à l'aide de réseaux
écologiques. Les différentes propriétés de ces réseaux peuvent fournir
des indications sur l'organisation des processus écologiques, mais aussi
sur la résilience du réseau face aux changements et aux perturbations
\citep{delmas2019analysing}. L'étude des réseaux écologique peut se
faire autant à l'échelle des écosystèmes qu'au niveau des espèces. Dans
le dernier cas, ils permettent d'observer le rôle et l'importance de
celles-ci à travers leurs interactions au sein du réseau
\citep{delmas2019analysing}. Plusieurs facteurs peuvent influencer ces
interactions tels que l'abondance locale des espèces, leurs traits ainsi
que les conditions environnementales \citep{delmas2019analysing}. En
tant qu'écologistes si bien formés par l'Université de Sherbrooke, nous
avons tenté de déterminer si les propriétés d'un réseau écologique sont
similaires à ceux d'un réseau d'interactions entre les étudiants du
cours BIO500 à l'hiver 2023. Au cours de leur parcours universitaire,
les étudiants ont été confrontés à réaliser de nombreux travaux
d'équipes qui représentent des collaborations avec différents individus.
Ces collaborations permettent d'analyser les interactions entre les
étudiants et les facteurs qui peuvent les influencer. L'objectif est
donc d'observer l'importance des étudiants au sein du réseau de
collaboration à l'aide de la centralité, mais aussi de comprendre
comment l'année du début du baccalauréat et la formation préalable
influencent cette centralité.

\hypertarget{muxe9thode}{%
\section{Méthode}\label{muxe9thode}}

\hypertarget{uxe9chantillonnage}{%
\subsection{Échantillonnage}\label{uxe9chantillonnage}}

Durant l'hiver 2023, les étudiants du cours de BIO500 ont été divisés en
10 équipes de 4 personnes. Chaque équipe a complété trois fichiers csv
concernant les cours auxquels ils ont réalisé des travaux d'équipes au
courant de leur baccalauréat. Le premier fichier permet identifier les
collaborations effectuées, c'est-à-dire les liens d'interactions avec
leurs différents coéquipiers en fonction du cours. Le deuxième fichier
représente des informations relatives aux étudiants de l'équipe. Plus
spécifiquement, leur région administrative, leur participation au régime
coopératif, leur année de début du baccalauréat, leur formation
préalable et leur numéro du programme. Puis, le dernier identifie les
cours dans lesquels des travaux ont été réalisés et d'autres données
pertinentes.

\hypertarget{traitement-et-analyse-des-donnuxe9es}{%
\subsection{Traitement et analyse des
données}\label{traitement-et-analyse-des-donnuxe9es}}

Une fois la compilation des données réalisée par chaque équipe, les
données ont été partagées à l'ensemble du groupe et celles-ci ont été
fusionnées à l'aide du logiciel R sous forme de trois fichiers csv. Par
la suite, les données des différents fichiers ont été nettoyées et
standardisées afin d'obtenir une conformité au sein des différentes
tables. Ces données ont ensuite été injectées à l'aide du logiciel R
dans le système de gestion de données SQLite3
\citep{muller_rsqlite_2023}. Ce système nous a permis d'extraire les
données d'intérêt via des requêtes qui sont nécessaires à la création
des figures. Il est à noter que ce n'est pas toutes les données qui ont
été utilisées lors des analyses. Les représentations visuelles du réseau
d'interaction, ainsi que la centralité ont été effectuées grâce au
package ``Igraph'', ainsi qu'à l'aide du package '' Scales '' du
locigiel R \citep{csardi_igraph_2023, wickham_scales_2022}. Le package
``Vioplot'' du locigiel R a aussi été utilisé pour la création du
diagramme en violon \citep{adler_vioplot_2022}. Ensuite, le package
``targets'' a été utilisé afin d'automatiser l'ensemble du processus et
d'augmenter la reproductibilité de cette étude
\citep{landau_tarchetypes_2023, landau_targets_2023}. Finalement, le
gabarit PNAS dans Rmarkdown a été utilisé grâce au package '' Rticles ''
\citep{allaire_rticles_2022}.

\hypertarget{ruxe9sultats}{%
\section{Résultats}\label{ruxe9sultats}}

\hypertarget{section}{%
\section{}\label{section}}

\begin{figure}
\centering
\includegraphics{rapport_files/figure-latex/unnamed-chunk-1-1.pdf}
\caption{\label{fig:plot1}Distribution de la centralité selon la session
de début de programme des individus du réseau. La taille des nœuds et
leur couleur varient selon la centralité des étudiants tel qu'expliqué
par la légende. L'épaisseur des connexions entre les nœuds reflètent
l'occurrence d'interactions entre deux étudiants. Plus les lignes sont
épaisses, plus les étudiants ont collaboré un grand nombre de fois.}
\end{figure}

\hypertarget{section-1}{%
\section{}\label{section-1}}

La Figure \ref{fig:plot1} représente le réseau de collaborations des
individus du cours BIO500 donné à l'hiver 2023 et leurs collègues. On y
observe une variation au niveau de l'importance des étudiants grâce au
score de centralité. En effet, la taille des nœuds, qui représente
chaque étudiant, varie de manière à ce qu'elle augmente plus le score de
centralité est haut, soit proche de 1. Les individus de faible
centralité sont retrouvés en marge du réseau alors que les individus
présentant une forte centralité sont retrouvés plus au centre. On y
décèle également plusieurs occurrences d'interactions entre des
étudiants. Effectivement, des lignes plus épaisses sont dénotées dans le
réseau.

\hypertarget{section-2}{%
\section{}\label{section-2}}

\begin{figure}
\centering
\includegraphics{rapport_files/figure-latex/unnamed-chunk-2-1.pdf}
\caption{\label{fig:plot2}Distribution de la centralité selon la session
de début de programme des individus du réseau. Les points mauves dans le
graphique correspondent au score de centralité de chaque individu. Le
point blanc représente la médiane du groupe et les boites noires
correspond à l'intervalle entre les quartiles. La ligne noir plus mince
montre l'étendu où 95\% des données se situent. Les côtés montrent la
distribution des données.}
\end{figure}

\hypertarget{section-3}{%
\section{}\label{section-3}}

La Figure \ref{fig:plot2} permet de visualiser l'impact de l'année de
début sur la centralité des individus. Les étudiants NA sont les
étudiants qui ne sont pas le cours BIO500 à l'hiver 2023 qui ont
effectués des travaux d'équipes avec les étudiants du cours lors de leur
parcours. Il est a noté que pour les sessions hiver 2019, été 2021,
automne 2021, hiver 2022 et automne 2022, on y retrouve seulement un
étudiant ayant débuté son parcours. Ces années ne possèdent donc pas un
assez grand échantillonnage pour obtenir des résultats significatifs. On
observe que les individus ayant commencé à l'automne 2020 possèdent une
médiane similaire aux étudiants ayant commencé à l'hiver 2020. Les
étudiants de l'automne 2019 possèdent une médiane plus basse que
l'automne 2020 et l'hiver 2020. Les étudiants qui ne sont pas dans le
cours possèdent aussi une médiane plus faible que l'automne 2020 et
l'hiver 2020, mais elle est similaire aux étudiants de l'automne 2019.
Par contre, les différences ne sont pas nécessairement significatives
étant donné que les intervalles de 95\% s'entrecoupent.

\hypertarget{section-4}{%
\section{}\label{section-4}}

\begin{figure}
\centering
\includegraphics{rapport_files/figure-latex/unnamed-chunk-3-1.pdf}
\caption{\label{fig:plot3}Moyenne de centralité selon la formation
préalable des individus du réseau. Les barres d'erreurs correspondent
aux écart-types associés à la moyenne de chaque classe soit :
préuniversitaire, technique, universitaire et NA (individus hors cours
BIO500 hiver 2023).}
\end{figure}

\hypertarget{section-5}{%
\section{}\label{section-5}}

En ce qui concerne la Figure \ref{fig:plot3} elle illustre la moyenne de
centralité retrouvée pour chacune des formations préalables effectués
par les étudiants du cours de BIO500 de l'hiver 2023 par rapport aux
individus qui ne sont pas dans le cours. On observe très peu de
variation par rapport à la formation préalable au niveau des étudiants
au sein du cours, mais celle des étudiants hors du cours est très
faible. Cependant, les barres d'erreurs se chevauchent ce qui indique
que les différences entre les moyennes ne sont pas significatives.

\hypertarget{discussion}{%
\section{Discussion}\label{discussion}}

Les propriétés du réseau de collaboration (Figure \ref{fig:plot1})
peuvent être utilisées pour identifier l'importance des individus à
l'aide de la centralité. En effet, une grande centralité signifie que
les individus ont eu une grande diversité d'interaction, donc des
collaborations avec plusieurs personnes différentes. Inversement, les
personnes ayant une faible centralité sont celles qui ont eu des
interactions très peu diversifiées, donc des collaborations avec les
mêmes individus. Cependant, il faut garder en tête que la centralité est
influencée par le nombre de collaborations effectué par les étudiants.
Par conséquent, il est normal pour les individus hors cours BIO500 à
l'hiver 2023 d'avoir une très faible centralité. Les individus avec une
grande centralité (ceux en rouge au centre du réseau) peuvent être
comparés à des espèces ``clés de voûte'' d'un réseau écologique en
raison que leur disparition compromettrait la structure et le
fonctionnement d'un écosystème \citep{cagua2019keystoneness}. De plus,
cette différence dans la centralité peut aussi être comparée aux
individus généraliste et spécialiste d'un réseau écologique
\citep{cagua2019keystoneness}. Les individus qui ont une faible
centralité seraient considérés des spécialistes, tandis que les
individus avec une forte centralité seraient considérés des
généralistes. Ainsi, la centralité nous confirme que le réseau de
collaboration comporte des propriétés similaires à un réseau écologiste,
ce qui répond à notre premier objectif. Contrairement à ce qu'on aurait
pu s'attendre, l'année de début n'a pas réellement d'impact sur la
centralité (Figure \ref{fig:plot3}). L'automne 2020 représente la
majorité des étudiants du cours BIO500, ce qui signifie qu'il s'agit des
individus ayant fait la plus diversité d'interactions. Ainsi, on
s'attendait à ce qu'il ait une plus grande centralité. Toutefois, les
étudiants des autres années ont des centralités similaires aux étudiants
de l'automne 2020. Ceci peut être expliqué par un manque de données ou
simplement que l'année de début n'influence pas la centralité. Il serait
possible de le comparer ces individus à des migrants arrivant dans une
nouvelle population et qui finissent par s'intégrer parfaitement à la
population les rendant indissociables.

Comme l'année de début, la formation préalable n'a pas d'impact sur la
centralité (Figure \ref{fig:plot3}). Nous pensions que la majorité des
étudiants avait suivi une formation préuniversitaire et que ceux de
techniques allaient avoir tendance à travailler ensemble donc être des
individus spécialistes. Leur petit nombre et leur spécialisation
auraient pu diminuer leur centralité ce qui n'est pas le cas.
Contrairement à nos attentes, la formation préalable n'influence pas la
centralité, puisque les étudiants provenant de technique ont agi comme
des généralistes dans notre réseau. Il est intéressant de noter que la
centralité des étudiants qui sont à l'extérieur du réseau semble être
beaucoup plus faible comme il est possible de voir dans les Figures
\ref{fig:plot2} et \ref{fig:plot3}. Ce résultat est normal étant donné
que ces étudiants ont effectué très peu de collaborations dans le réseau
dû au fait qu'ils ne sont pas les individus visés par l'analyse, soit
les étudiants du cours de BIO500 à l'hiver 2023. Ils sont donc moins
importants que les étudiants qui sont considérés dans le réseau.

\hypertarget{conclusion}{%
\section{Conclusion}\label{conclusion}}

En conclusion, les étudiants du cours BIO500 à l'hiver 2023 ont un
réseau d'interaction similaire à un réseau écologique. En effet, la
centralité a permis d'identifier que les individus avec la plus grande
centralité agissent comme des espèces ``clés de voûte'' en plus d'être
considérés comme des généralistes. Toutefois, il n'a pas été possible
d'observer les facteurs qui influencent la centralité. La formation
préalable et l'année de début n'ont pas eu d'effet significatif sur la
centralité. Il serait intéressant de venir observer des variables qui
pourraient réellement impacter la dynamique de notre réseau et la
centralité des individus telles que la participation aux événements
sociaux.

\hypertarget{remerciement}{%
\section{Remerciement}\label{remerciement}}

Nous aimerions prendre le temps de remercier Hugo Morin Brassard et
Félix-Olivier Dufour qui nous ont grandement aidé lors de nos nombreuses
difficultés rencontrées principalement celles dans notre code R.

\renewcommand\refname{References}
\bibliography{mybibfile.bib}


\end{document}
