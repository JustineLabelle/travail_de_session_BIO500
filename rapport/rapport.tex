\documentclass[preprint, 3p,
authoryear]{elsarticle} %review=doublespace preprint=single 5p=2 column
%%% Begin My package additions %%%%%%%%%%%%%%%%%%%

\usepackage[hyphens]{url}

  \journal{An awesome journal} % Sets Journal name

\usepackage{lineno} % add

\usepackage{graphicx}
%%%%%%%%%%%%%%%% end my additions to header

\usepackage[T1]{fontenc}
\usepackage{lmodern}
\usepackage{amssymb,amsmath}
\usepackage{ifxetex,ifluatex}
\usepackage{fixltx2e} % provides \textsubscript
% use upquote if available, for straight quotes in verbatim environments
\IfFileExists{upquote.sty}{\usepackage{upquote}}{}
\ifnum 0\ifxetex 1\fi\ifluatex 1\fi=0 % if pdftex
  \usepackage[utf8]{inputenc}
\else % if luatex or xelatex
  \usepackage{fontspec}
  \ifxetex
    \usepackage{xltxtra,xunicode}
  \fi
  \defaultfontfeatures{Mapping=tex-text,Scale=MatchLowercase}
  \newcommand{\euro}{€}
\fi
% use microtype if available
\IfFileExists{microtype.sty}{\usepackage{microtype}}{}
\usepackage[]{natbib}
\bibliographystyle{plainnat}

\usepackage{graphicx}
\ifxetex
  \usepackage[setpagesize=false, % page size defined by xetex
              unicode=false, % unicode breaks when used with xetex
              xetex]{hyperref}
\else
  \usepackage[unicode=true]{hyperref}
\fi
\hypersetup{breaklinks=true,
            bookmarks=true,
            pdfauthor={},
            pdftitle={La centralité dans les réseaux de collaboration étudiant ; une comparaison des similitudes avec les réseaux écologiques},
            colorlinks=false,
            urlcolor=blue,
            linkcolor=magenta,
            pdfborder={0 0 0}}

\setcounter{secnumdepth}{0}
% Pandoc toggle for numbering sections (defaults to be off)
\setcounter{secnumdepth}{0}


% tightlist command for lists without linebreak
\providecommand{\tightlist}{%
  \setlength{\itemsep}{0pt}\setlength{\parskip}{0pt}}






\begin{document}


\begin{frontmatter}

  \title{La centralité dans les réseaux de collaboration étudiant ; une
comparaison des similitudes avec les réseaux écologiques}
    \author[]{Philippe Bourassa%
  %
  }
   \ead{Philippe.Bourassa4@USherbrooke.ca} 
    \author[]{Simon Bourgeois%
  %
  }
   \ead{Simon.Bourgeois@USherbrooke.ca} 
    \author[]{Justine Labelle%
  %
  }
   \ead{Justine.Labelle@USherbrooke.ca} 
    \author[]{Kayla Trempe-Kay%
  %
  }
   \ead{Kayla.Trempe-Kay@USherbrooke.ca} 
      \affiliation[]{Universite de Sherbrooke, Departement de biologie,
2500 Boulevard de l'Universite, Sherbrooke, Quebec, J1K 2R1}
    \cortext[cor1]{Corresponding author}
  
  \begin{abstract}
  This is the abstract.

  It consists of two paragraphs.
  \end{abstract}
    \begin{keyword}
    Réseau écologique \sep Collaboration \sep Centralité \sep 
    
  \end{keyword}
  
 \end{frontmatter}

\hypertarget{introduction}{%
\section{Introduction}\label{introduction}}

Les écologistes ont comme pratique l'étude des interactions entre les
espèces qui peut être traduite et analysée à l'aide de réseaux
écologiques. Les différentes propriétés de ces réseaux peuvent fournir
des indications sur l'organisation des processus écologiques, mais aussi
sur la résilience du réseau face aux changements et aux perturbations
\citep{delmas2019analysing}. L'étude des réseaux écologique peut se
faire autant à l'échelle des écosystèmes qu'au niveau des espèces. Dans
le dernier cas, ils permettent d'observer le rôle et l'importance de
celles-ci à travers leurs interactions au sein du réseau
\citep{delmas2019analysing}. Plusieurs facteurs peuvent influencer ces
interactions tels que l'abondance locale des espèces, leurs traits ainsi
que les conditions environnementales \citep{delmas2019analysing}. En
tant qu'écologistes si bien formés par l'Université de Sherbrooke, nous
avons tenté de déterminer si les propriétés d'un réseau écologique sont
similaires à ceux d'un réseau d'interactions entre les étudiants du
cours BIO500 à l'hiver 2023. Au cours de leur parcours universitaire,
les étudiants ont été confrontés à réaliser de nombreux travaux
d'équipes qui représentent des collaborations avec différents individus.
Ces collaborations permettent d'analyser les interactions entre les
étudiants et les facteurs qui peuvent les influencer. L'objectif est
donc d'observer l'importance des étudiants au sein du réseau de
collaboration à l'aide de la centralité, mais aussi de comprendre
comment l'année du début du baccalauréat et la formation préalable
influencent cette centralité.

\hypertarget{muxe9thode}{%
\section{Méthode}\label{muxe9thode}}

\hypertarget{uxe9chantillonnage}{%
\subsection{Échantillonnage}\label{uxe9chantillonnage}}

Durant l'hiver 2023, les étudiants du cours de BIO500 ont été divisés en
10 équipes de 4 personnes. Chaque équipe a complété trois fichiers csv
concernant les cours auxquels ils ont réalisé des travaux d'équipes au
courant de leur baccalauréat. Le premier fichier permet identifier les
collaborations effectuées, c'est-à-dire les liens d'interactions avec
leurs différents coéquipiers en fonction du cours. Le deuxième fichier
représente des informations relatives aux étudiants de l'équipe. Plus
spécifiquement, leur région administrative, leur participation au régime
coopératif, leur année de début du baccalauréat, leur formation
préalable et leur numéro du programme. Puis, le dernier identifie les
cours dans lesquels des travaux ont été réalisés et d'autres données
pertinentes.

\hypertarget{traitement-et-analyse-des-donnuxe9es}{%
\subsection{Traitement et analyse des
données}\label{traitement-et-analyse-des-donnuxe9es}}

Une fois la compilation des données réalisée par chaque équipe, les
données ont été partagées à l'ensemble du groupe et celles-ci ont été
fusionnées à l'aide du logiciel R sous forme de trois fichiers csv. Par
la suite, les données des différents fichiers ont été nettoyées et
standardisées afin d'obtenir une conformité au sein des différentes
tables. Ces données ont ensuite été injectées à l'aide du logiciel R
dans le système de gestion de données SQLite3
\citep{muller_rsqlite_2023}. Ce système nous a permis d'extraire les
données d'intérêt via des requêtes qui sont nécessaires à la création
des figures. Il est à noter que ce n'est pas toutes les données qui ont
été utilisées lors des analyses. Les représentations visuelles du réseau
d'interaction, ainsi que la centralité ont été effectuées grâce au
package ``Igraph'', ainsi qu'à l'aide du package '' Scales '' du
locigiel R \citep{csardi_igraph_2023, wickham_scales_2022}. Le package
``Vioplot'' du locigiel R a aussi été utilisé pour la création du
diagramme en violon \citep{adler_vioplot_2022}. Ensuite, le package
``targets'' a été utilisé afin d'automatiser l'ensemble du processus et
d'augmenter la reproductibilité de cette étude
\citep{landau_tarchetypes_2023, landau_targets_2023}. Finalement, le
gabarit PNAS dans Rmarkdown a été utilisé grâce au package '' Rticles ''
\citep{allaire_rticles_2022}.

\hypertarget{ruxe9sultats}{%
\section{Résultats}\label{ruxe9sultats}}

\begin{figure}
\centering
\includegraphics{rapport_files/figure-latex/unnamed-chunk-1-1.pdf}
\caption{\label{fig:plot1}Distribution de la centralité selon la session
de début de programme des individus du réseau. La taille des nœuds et
leur couleur varient selon la centralité des étudiants tel qu'expliqué
par la légende. L'épaisseur des connexions entre les nœuds reflètent
l'occurrence d'interactions entre deux étudiants. Plus les lignes sont
épaisses, plus les étudiants ont collaboré un grand nombre de fois.}
\end{figure}

La Figure \ref{fig:plot1} représente le réseau de collaborations des
individus du cours BIO500 donné à l'hiver 2023 et leurs collègues. On y
observe une variation au niveau de l'importance des étudiants grâce au
score de centralité. En effet, la taille des nœuds, qui représente
chaque étudiant, varie de manière à ce qu'elle augmente plus le score de
centralité est haut, soit proche de 1. Les individus de faible
centralité sont retrouvés en marge du réseau alors que les individus
présentant une forte centralité sont retrouvés plus au centre. On y
décèle également plusieurs occurrences d'interactions entre des
étudiants. Effectivement, des lignes plus épaisses sont dénotées dans le
réseau.

\begin{figure}
\centering
\includegraphics{rapport_files/figure-latex/unnamed-chunk-2-1.pdf}
\caption{\label{fig:plot2}Distribution de la centralité selon la session
de début de programme des individus du réseau. Les points mauves dans le
graphique correspondent au score de centralité de chaque individu. Le
point blanc représente la médiane du groupe et les boites noires
correspond à l'intervalle entre les quartiles. La ligne noir plus mince
montre l'étendu où 95\% des données se situent. Les côtés montrent la
distribution des données.}
\end{figure}

\begin{figure}
\centering
\includegraphics{rapport_files/figure-latex/unnamed-chunk-3-1.pdf}
\caption{\label{fig:plot2}Moyenne de centralité selon la formation
préalable des individus du réseau. Les barres d'erreurs correspondent
aux écart-types associés à la moyenne de chaque classe soit :
préuniversitaire, technique, universitaire et NA (individus hors cours
BIO500 hiver 2023).}
\end{figure}

\hypertarget{discussion}{%
\section{Discussion}\label{discussion}}

Salut

\#Conclusion

\hypertarget{remerciement}{%
\section{Remerciement}\label{remerciement}}

Nous aimerions prendre le temps de remercier Hugo Morin Brassard et
Félix-Olivier Dufour qui nous ont grandement aidé lors de nos nombreuses
difficultés rencontrées principalement celles dans notre code R.

\renewcommand\refname{References}
\bibliography{mybibfile.bib}


\end{document}
