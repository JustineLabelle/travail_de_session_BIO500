\documentclass[9pt,twocolumn,twoside,]{pnas-new}

% Use the lineno option to display guide line numbers if required.
% Note that the use of elements such as single-column equations
% may affect the guide line number alignment.


\usepackage[T1]{fontenc}
\usepackage[utf8]{inputenc}
% Pandoc syntax highlighting
\usepackage{color}
\usepackage{fancyvrb}
\newcommand{\VerbBar}{|}
\newcommand{\VERB}{\Verb[commandchars=\\\{\}]}
\DefineVerbatimEnvironment{Highlighting}{Verbatim}{commandchars=\\\{\}}
% Add ',fontsize=\small' for more characters per line
\usepackage{framed}
\definecolor{shadecolor}{RGB}{248,248,248}
\newenvironment{Shaded}{\begin{snugshade}}{\end{snugshade}}
\newcommand{\AlertTok}[1]{\textcolor[rgb]{0.94,0.16,0.16}{#1}}
\newcommand{\AnnotationTok}[1]{\textcolor[rgb]{0.56,0.35,0.01}{\textbf{\textit{#1}}}}
\newcommand{\AttributeTok}[1]{\textcolor[rgb]{0.77,0.63,0.00}{#1}}
\newcommand{\BaseNTok}[1]{\textcolor[rgb]{0.00,0.00,0.81}{#1}}
\newcommand{\BuiltInTok}[1]{#1}
\newcommand{\CharTok}[1]{\textcolor[rgb]{0.31,0.60,0.02}{#1}}
\newcommand{\CommentTok}[1]{\textcolor[rgb]{0.56,0.35,0.01}{\textit{#1}}}
\newcommand{\CommentVarTok}[1]{\textcolor[rgb]{0.56,0.35,0.01}{\textbf{\textit{#1}}}}
\newcommand{\ConstantTok}[1]{\textcolor[rgb]{0.00,0.00,0.00}{#1}}
\newcommand{\ControlFlowTok}[1]{\textcolor[rgb]{0.13,0.29,0.53}{\textbf{#1}}}
\newcommand{\DataTypeTok}[1]{\textcolor[rgb]{0.13,0.29,0.53}{#1}}
\newcommand{\DecValTok}[1]{\textcolor[rgb]{0.00,0.00,0.81}{#1}}
\newcommand{\DocumentationTok}[1]{\textcolor[rgb]{0.56,0.35,0.01}{\textbf{\textit{#1}}}}
\newcommand{\ErrorTok}[1]{\textcolor[rgb]{0.64,0.00,0.00}{\textbf{#1}}}
\newcommand{\ExtensionTok}[1]{#1}
\newcommand{\FloatTok}[1]{\textcolor[rgb]{0.00,0.00,0.81}{#1}}
\newcommand{\FunctionTok}[1]{\textcolor[rgb]{0.00,0.00,0.00}{#1}}
\newcommand{\ImportTok}[1]{#1}
\newcommand{\InformationTok}[1]{\textcolor[rgb]{0.56,0.35,0.01}{\textbf{\textit{#1}}}}
\newcommand{\KeywordTok}[1]{\textcolor[rgb]{0.13,0.29,0.53}{\textbf{#1}}}
\newcommand{\NormalTok}[1]{#1}
\newcommand{\OperatorTok}[1]{\textcolor[rgb]{0.81,0.36,0.00}{\textbf{#1}}}
\newcommand{\OtherTok}[1]{\textcolor[rgb]{0.56,0.35,0.01}{#1}}
\newcommand{\PreprocessorTok}[1]{\textcolor[rgb]{0.56,0.35,0.01}{\textit{#1}}}
\newcommand{\RegionMarkerTok}[1]{#1}
\newcommand{\SpecialCharTok}[1]{\textcolor[rgb]{0.00,0.00,0.00}{#1}}
\newcommand{\SpecialStringTok}[1]{\textcolor[rgb]{0.31,0.60,0.02}{#1}}
\newcommand{\StringTok}[1]{\textcolor[rgb]{0.31,0.60,0.02}{#1}}
\newcommand{\VariableTok}[1]{\textcolor[rgb]{0.00,0.00,0.00}{#1}}
\newcommand{\VerbatimStringTok}[1]{\textcolor[rgb]{0.31,0.60,0.02}{#1}}
\newcommand{\WarningTok}[1]{\textcolor[rgb]{0.56,0.35,0.01}{\textbf{\textit{#1}}}}

% tightlist command for lists without linebreak
\providecommand{\tightlist}{%
  \setlength{\itemsep}{0pt}\setlength{\parskip}{0pt}}


% Pandoc citation processing
\newlength{\cslhangindent}
\setlength{\cslhangindent}{1.5em}
\newlength{\csllabelwidth}
\setlength{\csllabelwidth}{3em}
\newlength{\cslentryspacingunit} % times entry-spacing
\setlength{\cslentryspacingunit}{\parskip}
% for Pandoc 2.8 to 2.10.1
\newenvironment{cslreferences}%
  {}%
  {\par}
% For Pandoc 2.11+
\newenvironment{CSLReferences}[2] % #1 hanging-ident, #2 entry spacing
 {% don't indent paragraphs
  \setlength{\parindent}{0pt}
  % turn on hanging indent if param 1 is 1
  \ifodd #1
  \let\oldpar\par
  \def\par{\hangindent=\cslhangindent\oldpar}
  \fi
  % set entry spacing
  \setlength{\parskip}{#2\cslentryspacingunit}
 }%
 {}
\usepackage{calc}
\newcommand{\CSLBlock}[1]{#1\hfill\break}
\newcommand{\CSLLeftMargin}[1]{\parbox[t]{\csllabelwidth}{#1}}
\newcommand{\CSLRightInline}[1]{\parbox[t]{\linewidth - \csllabelwidth}{#1}\break}
\newcommand{\CSLIndent}[1]{\hspace{\cslhangindent}#1}


\templatetype{pnasresearcharticle}  % Choose template

\title{Template for preparing your research report submission to PNAS
using RMarkdown}

\author[a]{Simon Bourgeois}
\author[a]{Philippe Bourassa}
\author[a]{Kayla Trempe Kay}
\author[a]{Justine Labelle}

  \affil[a]{Université de Sherbrooke, Départment de biologie, 2500
Boulevard de l'Université, Sherbrooke, Québec, J1K 2R1}


% Please give the surname of the lead author for the running footer
\leadauthor{}

% Please add here a significance statement to explain the relevance of your work
\significancestatement{}


\authorcontributions{}



\correspondingauthor{\textsuperscript{} }

% Keywords are not mandatory, but authors are strongly encouraged to provide them. If provided, please include two to five keywords, separated by the pipe symbol, e.g:
 \keywords{  Réseau
écologique |  Centralité |  optional |  optional |  optional  } 

\begin{abstract}
Please provide an abstract of no more than 250 words in a single
paragraph. Abstracts should explain to the general reader the major
contributions of the article. References in the abstract must be cited
in full within the abstract itself and cited in the text.
\end{abstract}

\dates{This manuscript was compiled on \today}
\doi{\url{www.pnas.org/cgi/doi/10.1073/pnas.XXXXXXXXXX}}

\begin{document}

% Optional adjustment to line up main text (after abstract) of first page with line numbers, when using both lineno and twocolumn options.
% You should only change this length when you've finalised the article contents.
\verticaladjustment{-2pt}



\maketitle
\thispagestyle{firststyle}
\ifthenelse{\boolean{shortarticle}}{\ifthenelse{\boolean{singlecolumn}}{\abscontentformatted}{\abscontent}}{}

% If your first paragraph (i.e. with the \dropcap) contains a list environment (quote, quotation, theorem, definition, enumerate, itemize...), the line after the list may have some extra indentation. If this is the case, add \parshape=0 to the end of the list environment.

\acknow{}

\hypertarget{introduction}{%
\section{Introduction}\label{introduction}}

Les écologistes ont comme pratique l'étude des interactions entre les
espèces qui peut être traduite et analysée à l'aide de réseaux
écologiques. Les différentes propriétés de ces réseaux peuvent fournir
des indications sur l'organisation des processus écologiques, mais aussi
sur la résilience du réseau face aux changements et aux perturbations
(1). L'étude des réseaux écologique peut se faire autant à l'échelle des
écosystèmes qu'au niveau des espèces. Dans le dernier cas, ils
permettent d'observer le rôle et l'importance de celles-ci à travers
leurs interactions au sein du réseau (1). Plusieurs facteurs peuvent
influencer ces interactions tels que l'abondance locale des espèces,
leurs traits ainsi que les conditions environnementales (1). En tant
qu'écologistes si bien formés par l'Université de Sherbrooke, nous avons
tenté de déterminer si les propriétés d'un réseau écologique sont
similaires à ceux d'un réseau d'interactions entre les étudiants du
cours BIO500 à l'hiver 2023. Au cours de leur parcours universitaire,
les étudiants ont été confrontés à réaliser de nombreux travaux
d'équipes qui représentent des collaborations avec différents individus.
Ces collaborations permettent d'analyser les interactions entre les
étudiants et les facteurs qui peuvent les influencer. L'objectif est
donc d'observer l'importance des étudiants au sein du réseau de
collaboration à l'aide de la centralité, mais aussi de comprendre
comment l'année du début du baccalauréat et la formation préalable
influencent cette centralité.

\hypertarget{muxe9thode}{%
\section{Méthode}\label{muxe9thode}}

\begin{Shaded}
\begin{Highlighting}[]
\NormalTok{c}\OtherTok{\textless{}{-}} \FunctionTok{seq}\NormalTok{(}\DecValTok{1}\NormalTok{,}\DecValTok{5}\NormalTok{, }\AttributeTok{by=}\DecValTok{1}\NormalTok{)}
\NormalTok{c}
\end{Highlighting}
\end{Shaded}

\begin{verbatim}
## [1] 1 2 3 4 5
\end{verbatim}

\hypertarget{uxe9chantillonnage}{%
\subsection{Échantillonnage}\label{uxe9chantillonnage}}

Durant l'hiver 2023, les étudiants du cours de BIO500 ont été divisés en
10 équipes de 4 personnes. Chaque équipe a complété trois fichiers csv
concernant les cours auxquels ils ont réalisé des travaux d'équipes au
courant de leur baccalauréat. Le premier fichier permet identifier les
collaborations effectuées, c'est-à-dire les liens d'interactions avec
leurs différents coéquipiers en fonction du cours. Le deuxième fichier
représente des informations relatives aux étudiants de l'équipe. Plus
spécifiquement, leur région administrative, leur participation au régime
coopératif, leur année de début du baccalauréat, leur formation
préalable et leur numéro du programme. Puis, le dernier identifie les
cours dans lesquels des travaux ont été réalisés et d'autres données
pertinentes.

\hypertarget{traitement-et-analyse-des-donnuxe9es}{%
\subsection{Traitement et analyse des
données}\label{traitement-et-analyse-des-donnuxe9es}}

Une fois la compilation des données réalisée par chaque équipe, les
données ont été partagées à l'ensemble du groupe et celles-ci ont été
fusionnées à l'aide du logiciel R sous forme de trois fichiers csv. Par
la suite, les données des différents fichiers ont été nettoyées et
standardisées afin d'obtenir une conformité au sein des différentes
tables. Ces données ont ensuite été injectées à l'aide du logiciel R
dans le système de gestion de données SQLite3. Ce système nous a permis
d'extraire les données d'intérêt via des requêtes qui sont nécessaires à
la création des figures. Il est à noter que ce n'est pas toutes les
données qui ont été utilisées lors des analyses. Les représentations
visuelles du réseau d'interaction, ainsi que la centralité ont été
effectuées grâce au package ``Igraph'' du locigiel R. Le package
``Vioplot'' du locigiel R a aussi été utilisé pour la création du
diagramme en violon. Ensuite, le package ``targets'' a été utilisé afin
d'automatiser l'ensemble du processus et d'augmenter la reproductibilité
de cette étude. Finalement, le gabarit PNAS dans Rmarkdown a été utilisé
grâce au package '' Rticles ``.

\hypertarget{ruxe9sultats}{%
\section{Résultats}\label{ruxe9sultats}}

\hypertarget{discussion}{%
\section{Discussion}\label{discussion}}

\hypertarget{conclusion}{%
\section{Conclusion}\label{conclusion}}

\hypertarget{bibliographie}{%
\section*{Bibliographie}\label{bibliographie}}
\addcontentsline{toc}{section}{Bibliographie}

\showmatmethods
\showacknow
\pnasbreak

\hypertarget{refs}{}
\begin{CSLReferences}{0}{0}
\leavevmode\vadjust pre{\hypertarget{ref-delmas2019analysing}{}}%
\CSLLeftMargin{1. }%
\CSLRightInline{Delmas E, et al. (2019) Analysing ecological networks of
species interactions. \emph{Biological Reviews} 94(1):16--36.}

\end{CSLReferences}



% Bibliography
% \bibliography{pnas-sample}

\end{document}
