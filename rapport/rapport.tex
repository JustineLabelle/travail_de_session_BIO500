\documentclass[preprint, 3p,
authoryear]{elsarticle} %review=doublespace preprint=single 5p=2 column
%%% Begin My package additions %%%%%%%%%%%%%%%%%%%

\usepackage[hyphens]{url}

  \journal{An awesome journal} % Sets Journal name

\usepackage{lineno} % add

\usepackage{graphicx}
%%%%%%%%%%%%%%%% end my additions to header

\usepackage[T1]{fontenc}
\usepackage{lmodern}
\usepackage{amssymb,amsmath}
\usepackage{ifxetex,ifluatex}
\usepackage{fixltx2e} % provides \textsubscript
% use upquote if available, for straight quotes in verbatim environments
\IfFileExists{upquote.sty}{\usepackage{upquote}}{}
\ifnum 0\ifxetex 1\fi\ifluatex 1\fi=0 % if pdftex
  \usepackage[utf8]{inputenc}
\else % if luatex or xelatex
  \usepackage{fontspec}
  \ifxetex
    \usepackage{xltxtra,xunicode}
  \fi
  \defaultfontfeatures{Mapping=tex-text,Scale=MatchLowercase}
  \newcommand{\euro}{€}
\fi
% use microtype if available
\IfFileExists{microtype.sty}{\usepackage{microtype}}{}
\usepackage[]{natbib}
\bibliographystyle{plainnat}

\usepackage{graphicx}
\ifxetex
  \usepackage[setpagesize=false, % page size defined by xetex
              unicode=false, % unicode breaks when used with xetex
              xetex]{hyperref}
\else
  \usepackage[unicode=true]{hyperref}
\fi
\hypersetup{breaklinks=true,
            bookmarks=true,
            pdfauthor={},
            pdftitle={Short Paper},
            colorlinks=false,
            urlcolor=blue,
            linkcolor=magenta,
            pdfborder={0 0 0}}

\setcounter{secnumdepth}{0}
% Pandoc toggle for numbering sections (defaults to be off)
\setcounter{secnumdepth}{0}


% tightlist command for lists without linebreak
\providecommand{\tightlist}{%
  \setlength{\itemsep}{0pt}\setlength{\parskip}{0pt}}






\begin{document}


\begin{frontmatter}

  \title{Short Paper}
    \author[]{Philippe Bourassa%
  %
  }
   \ead{Philippe.Bourassa4@USherbrooke.ca} 
    \author[]{Simon Bourgeois%
  %
  }
   \ead{Simon.Bourgeois@USherbrooke.ca} 
    \author[]{Justine Labelle%
  %
  }
   \ead{Justine.Labelle@USherbrooke.ca} 
    \author[]{Kayla Trempe-Kay%
  %
  }
   \ead{Kayla.Trempe-Kay@USherbrooke.ca} 
      \affiliation[]{Université de Sherbrooke, Département de biologie,
2500 Boulevard de l'Université, Sherbrooke, Québec, J1K 2R1}
    \cortext[cor1]{Corresponding author}
  
  \begin{abstract}
  This is the abstract.

  It consists of two paragraphs.
  \end{abstract}
    \begin{keyword}
    Réseau écologique \sep Centralité \sep Réseau écologique \sep 
    Centralité
  \end{keyword}
  
 \end{frontmatter}

\hypertarget{introduction}{%
\section{Introduction}\label{introduction}}

Les écologistes ont comme pratique l'étude des interactions entre les
espèces qui peut être traduite et analysée à l'aide de réseaux
écologiques. Les différentes propriétés de ces réseaux peuvent fournir
des indications sur l'organisation des processus écologiques, mais aussi
sur la résilience du réseau face aux changements et aux perturbations
\citet{delmas2019analysing}. L'étude des réseaux écologique peut se
faire autant à l'échelle des écosystèmes qu'au niveau des espèces. Dans
le dernier cas, ils permettent d'observer le rôle et l'importance de
celles-ci à travers leurs interactions au sein du réseau
\citet{delmas2019analysing}. Plusieurs facteurs peuvent influencer ces
interactions tels que l'abondance locale des espèces, leurs traits ainsi
que les conditions environnementales \citet{delmas2019analysing}. En
tant qu'écologistes si bien formés par l'Université de Sherbrooke, nous
avons tenté de déterminer si les propriétés d'un réseau écologique sont
similaires à ceux d'un réseau d'interactions entre les étudiants du
cours BIO500 à l'hiver 2023. Au cours de leur parcours universitaire,
les étudiants ont été confrontés à réaliser de nombreux travaux
d'équipes qui représentent des collaborations avec différents individus.
Ces collaborations permettent d'analyser les interactions entre les
étudiants et les facteurs qui peuvent les influencer. L'objectif est
donc d'observer l'importance des étudiants au sein du réseau de
collaboration à l'aide de la centralité, mais aussi de comprendre
comment l'année du début du baccalauréat et la formation préalable
influencent cette centralité.

\hypertarget{ruxe9sultats}{%
\section{Résultats}\label{ruxe9sultats}}

\includegraphics{rapport_files/figure-latex/unnamed-chunk-1-1.pdf}

\textbf{Figure 1} : Réseau de collaborations des étudiants du cours
BIO500 à l'hiver 2023.

\includegraphics{rapport_files/figure-latex/unnamed-chunk-2-1.pdf}
\textbf{Figure 2} : Distribution de la centralité selon la session de
début de programme.

\includegraphics{rapport_files/figure-latex/unnamed-chunk-3-1.pdf}
\textbf{Figure 3} : Moyenne de centralité par formation préalable.

\hypertarget{discussion}{%
\section{Discussion}\label{discussion}}

Salut

\renewcommand\refname{References}
\bibliography{mybibfile.bib}


\end{document}
